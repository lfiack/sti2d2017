\documentclass[pdftex,a4paper,12pt]{article}

\usepackage[francais]{babel}
\usepackage[utf8x]{inputenc}
\usepackage{mathptmx}
\usepackage{geometry}
\geometry{a4paper,headheight=16pt,tmargin=20mm,
          bmargin=20mm,lmargin=20mm,rmargin=20mm}
\thispagestyle{empty}
\begin{document}
\textbf{Test de 3 minutes}
\begin{enumerate}
	\item Lisez l'intégralité du sujet avant de commencer
	\item Notez votre nom en haut à droite de cette page
	\item Entourez le mot \og{}nom\fg{} dans la question deux
	\item Dessinez cinq petits carrés dans le coin supérieur droit sous votre nom
	\item Mettez une croix dans chaque carré que vous venez de dessiner
	\item Mettez un cercle autour de chaque carré
	\item Signez en bas de cette page
	\item Après le titre, écrivez \og{}Oui Oui Oui\fg{}
	\item Soulignez les questions sept et huit
	\item Mettez un X dans le coin inférieur gauche de cette page
	\item Dessinez un triangle autour du X que vous venez de dessiner
	\item De l'autre côté de cette page, multipliez soixante-dix par trente
	\item Entourez le mot \og{}coin\fg{} dans la question quatre
	\item Prononcez distinctement et à voix haute votre prénom quand vous arrivez à cette question
	\item Si vous pensez avoir correctement suivi les instructions, dites à votre voisin \og{}J'ai correctement suivi les instructions\fg{}
	\item De l'autre côté de cette feuille, additionnez 107 et 278
	\item Comptez à voix haute de un à dix
	\item Si vous êtes la première personne à arriver à cette question, écrivez votre nom au tableau
	\item Faites trois petits trous avec votre stylo ici . . .
	\item Maintenant que vous avez lu l'intégralité du sujet, ne faites que les questions une et deux
\end{enumerate}
\end{document}
