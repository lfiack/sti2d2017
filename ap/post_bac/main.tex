\documentclass[miniframe]{lbpbeamer}
%\documentclass[handout]{lbpbeamer}
%%%%%%%%%%%%% option de la classe lbpbeamer.cls
% toutes les options standards de la classe beamer
% le type d'en-tete pour l'appel des sections : 
%  - default : currentsection | currentsubsection
%  - miniframe : sections + puces
%  - tree : sections + currentsubsection
%  - split : sections + subsections
%%%%%%%%%%%%%%%%%%%%%%%%%%%%%%%%%%%%%%%%%%%%%%%%%%

%%%%%%%%%%%%% appel des packages
\usepackage[english,french]{babel}
\usepackage{epsfig}
\usepackage[T1]{fontenc}
\usepackage[utf8]{inputenc}
\usepackage{lmodern} 
\usepackage{times}
\usepackage{multirow}
\usepackage{animate}
\usepackage{hyperref}
\usepackage{movie15}
\usepackage{wasysym}
\usepackage[squaren, Gray, cdot]{SIunits}
\usepackage{tikz}
\usetikzlibrary{calc,trees,positioning,arrows,chains,shapes.geometric,%
	decorations.pathreplacing,decorations.pathmorphing,shapes,%
matrix,shapes.symbols,plotmarks,decorations.markings,shadows}

\usepackage{tabularx}
\newcolumntype{L}[1]{>{\raggedright\let\newline\\\arraybackslash\hspace{0pt}}m{#1}}
\newcolumntype{C}[1]{>{\centering\let\newline\\\arraybackslash\hspace{0pt}}m{#1}}
\newcolumntype{R}[1]{>{\raggedleft\let\newline\\\arraybackslash\hspace{0pt}}m{#1}}

\newcommand{\backupbegin}{
	\newcounter{framenumberappendix}
	\setcounter{framenumberappendix}{\value{framenumber}}
}
\newcommand{\backupend}{
	\addtocounter{framenumberappendix}{-\value{framenumber}}
	\addtocounter{framenumber}{\value{framenumberappendix}} 
}

%\usepackage{enumitem}

%%%%%%%%%%%%%%%%%%%%%%%%%%%%%%%%%%%%%%%%%%%%%%%%%%
%\newlist{fleche}{itemize}{1}
%\setlist[fleche]{label=$\rightarrow$,font=\color{blue}}


%\newcommand{\smiley}{\tikz[baseline=-0.75ex,black]{
%		    \draw circle (2mm);
%			\node[fill,circle,inner sep=0.5pt] (left eye) at (135:0.8mm) {};
%			\node[fill,circle,inner sep=0.5pt] (right eye) at (45:0.8mm) {};
%			\draw (-145:0.9mm) arc (-120:-60:1.5mm);
%			    }
%			}

\newcommand{\orto}{^{\circ}}

%%%%%%%%%%%%% appel du plan a chaque subsection
%% en 1 colonne

%\AtBeginSection[]{
%	\frame{%<handout:0>{
%	\frametitle{Plan}
%  \begin{columns}[t]
%  \begin{column}{0.5\linewidth}
%  \tableofcontents[sections={1-3}, currentsection,subsectionstyle=show/show/shaded]
%  \end{column}
%  \begin{column}{0.5\linewidth}
%  \tableofcontents[sections={4-6}, currentsection,subsectionstyle=show/show/shaded]
%  \end{column}
%  \end{columns}
%  }
%}

%% ou en 2 colonnes s'il y a trop de sections
    
%\AtBeginSubsection[]{
%	\frame{%<handout:0>{
%	\frametitle{Summary}
%  \begin{columns}[t]
%  \begin{column}{0.5\linewidth}
%  \tableofcontents[sections={1-6},currentsection, subsectionstyle=show/shaded/hide]
%  \end{column}
%  \begin{column}{0.5\linewidth}
%  \tableofcontents[sections={7-12},currentsection,subsectionstyle=show/shaded/hide]
%  \end{column}
%  \end{columns}
%  }
%}	
%%%%%%%%%%%%%%%%%%%%%%%%%%%%%%%%%%%%%%%%%%%%%%%%%%	

%%%%%%%%%%%%% Pour supprimer les symboles de navigation
\setbeamertemplate{navigation symbols}{}
%%%%%%%%%%%%%%%%%%%%%%%%%%%%%%%%%%%%%%%%%%%%%%%%%%

%%%%%%%%%%%%% Personnalisation des theoremes
\newtheorem{theoreme}{Th\'eor\`eme}
\newtheorem{preuve}{D\'emonstration}
\newtheorem{define}{D\'efinition}
%%%%%%%%%%%%%%%%%%%%%%%%%%%%%%%%%%%%%%%%%%%%%%%%%%	
		
%%%%%%%%%%%%% Infos personnelles au document	
\title[Poursuites d'études]{Poursuites d'études}
\subtitle{Ou \og{}quoi que c'est qu'on fait après qu'on a le bac ?\fg{}}
\author[Fiack]{Laurent Fiack}
%\email{}
\institute[Lycée Blaise Pascal]{Lycée Blaise Pascal}
\date[\today]{\today}
\logo{}
%%%%%%%%%%%%%%%%%%%%%%%%%%%%%%%%%%%%%%%%%%%%%%%%%%

\newcommand{\figpath}{figures/}

\usepackage{scrextend}

\begin{document}

%%%%%%%%%%%%% Background des slides	
\usebackgroundtemplate{}
%% cet option permet d'ins\'erer une image en fond-ecran
%% la commande \usebackgroundtemplate{} permet de
%% supprimer le fond a partir du moment ou il est 
%% appele
%%%%%%%%%%%%%%%%%%%%%%%%%%%%%%%%%%%%%%%%%%%%%%%%%%	

%%%%%%%%%%%%% frame title
\frame{\titlepage}
\usebackgroundtemplate{}
\logo{}
%\logoheader{taille}{emplacement-image}
%% on supprime les logos des autres frames		
%%%%%%%%%%%%%%%%%%%%%%%%%%%%%%%%%%%%%%%%%%%%%%%%%%
\section[Post bac]{Poursuites d'études}
\subsection[Post bac]{Poursuites d'études}
\frame{
	\frametitle{Quelles sont les différentes orientations ?}
	\begin{itemize}
		\item BTS/DUT ;
			\begin{itemize}
				\item Licence Professionnelle,
				\item Prépa ATS,
				\item Certaines écoles d'ingénieurs ;
			\end{itemize}
		\item École d'ingénieurs post-bac ;
		\item Université (Licence/Master) ;
		\item CPGE (Classe Préparatoire aux Grandes Écoles) TSI (Technologie et Sciences Industrielles)
			\begin{itemize}
				\item École d'ingénieurs.
			\end{itemize}
	\end{itemize}
}

\frame{
	\frametitle{BTS : Brevet de technicien supérieur}
	Assez proche du DUT. Informations disponibles \texttt{\href{http://www.onisep.fr/Choisir-mes-etudes/Apres-le-bac/Que-faire-apres-le-bac/Que-faire-apres-le-bac-STI2D-sciences-et-technologies-de-l-industrie-et-du-developpement-durable/Les-BTS-et-DUT-apres-le-bac-STI2D}{ici}}.
	\begin{itemize}
		\item Formation professionnalisante \textbf{spécialisée} ;
		\item Formation accessible sur dossier ;
		\item Se déroule dans un \textbf{lycée}, organisation proche du lycée ;
		\item Études courtes : diplôme en 2 ans ;
		\item Poursuite possible en licence pro (1 an) ; 
		\item Possibilité d'une année de prépa ATS (1an) pour entrer en école d'ingénieurs (3 années supplémentaires).
	\end{itemize}
}

\frame{
	\frametitle{Exemples de BTS}
	Toutes les formations ne sont pas disponibles dans tous les lycées. Il faut se renseigner. Voici quelques thématiques :
	\begin{itemize}
		\item Audiovisuel, informatique, télécoms et numérique ;
		\item Bâtiments, travaux publics, architecture ;
		\item Commerce ;
		\item Construction navale, ferroviaire et aéronautique, maintenance, matériaux et mécanique ;
		\item Énergies, électronique et environnement ;
		\item Paramédical.
	\end{itemize}
}

\frame{
	\frametitle{DUT : Diplôme universitaire de technologie}
	Quelques similitudes avec le BTS.
	Informations disponibles \texttt{\href{http://www.onisep.fr/Choisir-mes-etudes/Apres-le-bac/Que-faire-apres-le-bac/Que-faire-apres-le-bac-STI2D-sciences-et-technologies-de-l-industrie-et-du-developpement-durable/Les-BTS-et-DUT-apres-le-bac-STI2D}{ici}}.
	\begin{itemize}
		\item Formation professionnalisante \textbf{généraliste} ;
		\item Formation accessible sur dossier ;
		\item Formation \textbf{universitaire}, se déroule dans un \textbf{IUT} (Institut Universitaire de Technologie), 
			l'organisation est \textbf{assez différente du lycée}.
		\item Études courtes : diplôme en 2 ans. Comme le DUT est plus généraliste que le BTS, il y a \textbf{plus de possibilités de poursuite d'études}.
		\item Poursuite possible en licence pro (1 an) ou \textbf{dans certaines écoles d'ingénieurs} (3 ans)
	\end{itemize}
	
}

\frame{
	\frametitle{Exemples de DUT}
	DUT accessibles avec un bac STI2D:\\
\vspace{1em}
	\scriptsize
	\begin{minipage}[b]{0.48\linewidth}
		\begin{itemize}
			\item Chimie option chimie des matériaux ;
			\item Génie chimique, génie des procédés options bio-procédés (GC GP) et procédés (GCh GP) ;
			\item Génie civil -- construction durable ;
			\item Génie électrique et informatique industrielle (GEII) ;
			\item Génie industriel et maintenance (GIM) ;
			\item Génie mécanique et productique (GMP) ;
			\item Génie thermique et énergie (GTE) ;
			\item Hygiène, sécurité, environnement ;
		\end{itemize}
	\end{minipage}
	\hfill
	\begin{minipage}[b]{0.48\linewidth}
		\begin{itemize}
			\item Informatique ;
			\item Mesures physiques (MP) ;
			\item Métiers du multimédia et de l'Internet (MMI) ;
			\item Packaging, emballage et conditionnement (PEC) ;
			\item Qualité, logistique industrielle et organisation (QLIO) ;
			\item Réseaux et télécommunications (RT) ;
			\item Science et génie des matériaux ;
			\item Statistique et informatique décisionnelle (STID).
		\end{itemize}
	\end{minipage}
}

\frame{
	\frametitle{Exemple d'emploi du temps en DUT GEII}
	\begin{itemize}
		\item Exemple: module d'informatique au premier semestre de la première année ;
			\begin{itemize}
				\item Du 31 octobre au 20 janvier ;
			\end{itemize}
		\item Promo d'environ 100 étudiants ;
			\begin{itemize}
				\item Découpée en 4 groupes de TD,
				\item Eux mêmes découpés en 2 groupes de TP ;
			\end{itemize}
		\item Module organisé en CM/TD/TP ;
		\item Plusieurs modules en parallèle ;
			\begin{itemize}
				\item On finit le module SIN pendant qu'on commence le module Informatique ;
			\end{itemize}
		\item Une fois le module terminé, on passe à autre chose ;
			\begin{itemize}
				\item On continue à faire de l'informatique, avec le module microprocesseur au S2.
			\end{itemize}
	\end{itemize}
		
	\texttt{Exemple en PDF}
	% TODO mail Tahar
}

\frame{
	\frametitle{Écoles d'ingénieurs post-bac}
	Alors que les programmes en BTS/DUT sont définis par le ministère, les écoles d'ingénieurs sont toutes différentes les unes des autres.
	Informations disponibles \texttt{\href{http://www.onisep.fr/Choisir-mes-etudes/Apres-le-bac/Que-faire-apres-le-bac/Que-faire-apres-le-bac-STI2D-sciences-et-technologies-de-l-industrie-et-du-developpement-durable/Les-ecoles-apres-le-bac-STI2D}{ici}}.
	\begin{itemize}
		\item Formation professionnalisante spécialisée ou généraliste, possibilité de personnaliser le cursus ;
		\item Formation accessible sur dossier, concours ET entretiens ;
		\item Bon niveau en maths, en physique et en ETT requis ;
		\item La formation varie d'une école à l'autre, mais l'organisation ressemble plus à celle du DUT ;
		\item Études longues : diplôme en 5 ans. Prépa intégré ou \og{}diluée\fg{} ;
		\item Possibilité de faire un master recherche pour continuer en doctorat.
	\end{itemize}

	\texttt{\href{http://www.geipi-polytech.org/les-ecoles-du-concours-geipi-polytech}{Exemple de \og{}banque\fg{} de concours}}.
%	\texttt{\href{http://www.geipi-polytech.org/les-ecoles-du-concours-geipi-polytech#}{test}}.
}

\frame{
	\frametitle{L'université}
	Informations disponibles \texttt{\href{http://www.onisep.fr/Choisir-mes-etudes/Apres-le-bac/Que-faire-apres-le-bac/Que-faire-apres-le-bac-STI2D-sciences-et-technologies-de-l-industrie-et-du-developpement-durable/L-universite-apres-le-bac-STI2D}{ici}}.
	\begin{itemize}
		\item Formation théorique généraliste, puis spécialisation possible ;
		\item Ouverte à tous (mais ça risque de changer) ;
		\item Bon niveau requis pour réussir, et surtout \textbf{beaucoup d'autonomie} ;
		\item Oranisation universitaire, tronc commun et UE spécialisées ;
		\item Études plutôt longues, 3 ans pour la licence, 5 pour le master ;
		\item Possibilité de poursuivre en doctorat.
	\end{itemize}
}

\frame{
	\frametitle{Prépa TSI}
	Informations disponibles \texttt{\href{http://www.onisep.fr/Choisir-mes-etudes/Apres-le-bac/Que-faire-apres-le-bac/Que-faire-apres-le-bac-STI2D-sciences-et-technologies-de-l-industrie-et-du-developpement-durable/Les-classes-preparatoires-aux-grandes-ecoles-apres-le-bac-STI2D}{ici}}.
	\begin{itemize}
		\item Formation \textbf{non-}professionnalisante. Ne délivre pas de diplôme, il \textbf{faut} continuer en école d'ingénieurs ;
		\item Formation accessible sur dossier, bon niveau requis ;
		\item Beaucoup de travail et de rigueur sont demandés pour réussir ;
		\item Se déroule au lycée, organisation proche de celle du lycée ;
		\item Dure 2 ans mais mène à des études plus longues (2 années de prépas puis 3 ans d'école d'ingénieurs).
	\end{itemize}
}

\end{document}
