\documentclass[miniframe]{lbpbeamer}
%\documentclass[handout]{lbpbeamer}
%%%%%%%%%%%%% option de la classe lbpbeamer.cls
% toutes les options standards de la classe beamer
% le type d'en-tete pour l'appel des sections : 
%  - default : currentsection | currentsubsection
%  - miniframe : sections + puces
%  - tree : sections + currentsubsection
%  - split : sections + subsections
%%%%%%%%%%%%%%%%%%%%%%%%%%%%%%%%%%%%%%%%%%%%%%%%%%

%%%%%%%%%%%%% appel des packages
\usepackage[english,french]{babel}
\usepackage{epsfig}
\usepackage[T1]{fontenc}
\usepackage[utf8]{inputenc}
\usepackage{lmodern} 
\usepackage{times}
\usepackage{multirow}
\usepackage{animate}
\usepackage{hyperref}
\usepackage{movie15}
\usepackage{wasysym}
\usepackage[squaren, Gray, cdot]{SIunits}
\usepackage{tikz}
\usetikzlibrary{calc,trees,positioning,arrows,chains,shapes.geometric,%
	decorations.pathreplacing,decorations.pathmorphing,shapes,%
matrix,shapes.symbols,plotmarks,decorations.markings,shadows}

\usepackage{tabularx}
\newcolumntype{L}[1]{>{\raggedright\let\newline\\\arraybackslash\hspace{0pt}}m{#1}}
\newcolumntype{C}[1]{>{\centering\let\newline\\\arraybackslash\hspace{0pt}}m{#1}}
\newcolumntype{R}[1]{>{\raggedleft\let\newline\\\arraybackslash\hspace{0pt}}m{#1}}

\newcommand{\backupbegin}{
	\newcounter{framenumberappendix}
	\setcounter{framenumberappendix}{\value{framenumber}}
}
\newcommand{\backupend}{
	\addtocounter{framenumberappendix}{-\value{framenumber}}
	\addtocounter{framenumber}{\value{framenumberappendix}} 
}

%\usepackage{enumitem}

%%%%%%%%%%%%%%%%%%%%%%%%%%%%%%%%%%%%%%%%%%%%%%%%%%
%\newlist{fleche}{itemize}{1}
%\setlist[fleche]{label=$\rightarrow$,font=\color{blue}}


%\newcommand{\smiley}{\tikz[baseline=-0.75ex,black]{
%		    \draw circle (2mm);
%			\node[fill,circle,inner sep=0.5pt] (left eye) at (135:0.8mm) {};
%			\node[fill,circle,inner sep=0.5pt] (right eye) at (45:0.8mm) {};
%			\draw (-145:0.9mm) arc (-120:-60:1.5mm);
%			    }
%			}

\newcommand{\orto}{^{\circ}}

%%%%%%%%%%%%% appel du plan a chaque subsection
%% en 1 colonne

%\AtBeginSection[]{
%	\frame{%<handout:0>{
%	\frametitle{Plan}
%  \begin{columns}[t]
%  \begin{column}{0.5\linewidth}
%  \tableofcontents[sections={1-3}, currentsection,subsectionstyle=show/show/shaded]
%  \end{column}
%  \begin{column}{0.5\linewidth}
%  \tableofcontents[sections={4-6}, currentsection,subsectionstyle=show/show/shaded]
%  \end{column}
%  \end{columns}
%  }
%}

%% ou en 2 colonnes s'il y a trop de sections
    
%\AtBeginSubsection[]{
%	\frame{%<handout:0>{
%	\frametitle{Summary}
%  \begin{columns}[t]
%  \begin{column}{0.5\linewidth}
%  \tableofcontents[sections={1-6},currentsection, subsectionstyle=show/shaded/hide]
%  \end{column}
%  \begin{column}{0.5\linewidth}
%  \tableofcontents[sections={7-12},currentsection,subsectionstyle=show/shaded/hide]
%  \end{column}
%  \end{columns}
%  }
%}	
%%%%%%%%%%%%%%%%%%%%%%%%%%%%%%%%%%%%%%%%%%%%%%%%%%	

%%%%%%%%%%%%% Pour supprimer les symboles de navigation
\setbeamertemplate{navigation symbols}{}
%%%%%%%%%%%%%%%%%%%%%%%%%%%%%%%%%%%%%%%%%%%%%%%%%%

%%%%%%%%%%%%% Personnalisation des theoremes
\newtheorem{theoreme}{Th\'eor\`eme}
\newtheorem{preuve}{D\'emonstration}
\newtheorem{define}{D\'efinition}
%%%%%%%%%%%%%%%%%%%%%%%%%%%%%%%%%%%%%%%%%%%%%%%%%%	
		
%%%%%%%%%%%%% Infos personnelles au document	
\title[Histoire des sciences]{Histoire des sciences}
\subtitle{Histoire des consoles de jeux vidéo}
\author[Fiack]{Laurent Fiack}
%\email{}
\institute[Lycée Blaise Pascal]{Lycée Blaise Pascal}
\date[\today]{\today}
\logo{}
%%%%%%%%%%%%%%%%%%%%%%%%%%%%%%%%%%%%%%%%%%%%%%%%%%

\newcommand{\figpath}{figures/}

\usepackage{scrextend}

\begin{document}

%%%%%%%%%%%%% Background des slides	
\usebackgroundtemplate{}
%% cet option permet d'ins\'erer une image en fond-ecran
%% la commande \usebackgroundtemplate{} permet de
%% supprimer le fond a partir du moment ou il est 
%% appele
%%%%%%%%%%%%%%%%%%%%%%%%%%%%%%%%%%%%%%%%%%%%%%%%%%	

%%%%%%%%%%%%% frame title
\frame{\titlepage}
\usebackgroundtemplate{}
\logo{}
%\logoheader{taille}{emplacement-image}
%% on supprime les logos des autres frames		
%%%%%%%%%%%%%%%%%%%%%%%%%%%%%%%%%%%%%%%%%%%%%%%%%%
\section[Premiers jeux vidéo]{Premiers jeux vidéo de l'histoire}
\subsection[Premiers jeux vidéo]{Premiers jeux vidéo de l'histoire}
\frame{
	\frametitle{Premier jeu vidéo de l'histoire ?}
	\pause
	\begin{itemize}
		\item \textbf{1947:} Cathode-ray tube amusement device
			\begin{itemize}
				\item Simulateur de tir de missile
				\item Pas sorti des labos
			\end{itemize}
	\end{itemize}
	\pause
	\centering
	\includegraphics[width=.7\linewidth]{./figures/Cathode_ray_tube_amusement_device_-_schematic.jpg}
}

\frame{
	\frametitle{Tube à vide}
	\begin{itemize}
		\item Ancêtre du transistor (inventé fin 1947)
		\item Amplificateur (analogique) ou interrupteur (numérique)
	\end{itemize}
	\centering
	\includegraphics[width=.9\linewidth]{./figures/1280px-Elektronenroehren-auswahl.jpg}
}

\frame{
	\frametitle{Premier jeu vidéo de l'histoire ?}
	\begin{itemize}
		\item \textbf{1958:} Tennis for two
			\begin{itemize}
				\item Entièrement analogique, se branche sur un oscilloscope
				\item Pas commercialisé
			\end{itemize}
	\end{itemize}
	\pause
	\centering
	\includegraphics[width=0.8\linewidth]{./figures/Tennis_for_Two_Machine_at_CAX_2010.jpg}
}

\frame{
	\frametitle{Tennis for two}
	\begin{itemize}
		\item Transistors et amplificateurs opérationnels
	\end{itemize}
	\centering
	\includegraphics[width=.8\linewidth]{./figures/Tennis_For_Two_Schematic_1.jpg}
}

\frame{
	\frametitle{Tennis for two}
	\begin{itemize}
		\item Transistors et amplificateurs opérationnels
	\end{itemize}
	\centering
	\includegraphics[width=.9\linewidth]{./figures/t42_03.jpg}
}

\frame{
	\frametitle{Premier jeu vidéo de l'histoire ?}
	\begin{minipage}[b]{0.48\linewidth}
		\begin{itemize}
			\item \textbf{1972:} Pong
				\begin{itemize}
					\item Borne d'arcade, 
					\item Numérique, mais non-programmable 
					\item (microprocesseur inventé en 1971)
					\item Naissance d'Atari
				\end{itemize}
		\end{itemize}
		\vspace{2cm}
		\pause
	\end{minipage}
	\hfill
	\begin{minipage}[b]{0.48\linewidth}
		\centering
		\includegraphics[width=.8\linewidth]{./figures/Atari_Pong_arcade_game_cabinet.jpg}
	\end{minipage}
}

\frame{
	\frametitle{Pong}
	\begin{itemize}
		\item Circuits logiques TTL
	\end{itemize}
	\centering
	\includegraphics[width=.9\linewidth]{./figures/atari_pong-doubles.png}
}

\frame{
	\frametitle{Pong}
	\begin{itemize}
		\item Circuits logiques TTL
	\end{itemize}
	\centering
	\includegraphics[width=.9\linewidth]{./figures/Back_of_the_arcade_cabinets_Pong.png}
}

\section[Première console]{Première console de salon}
\subsection[Première console]{Première console de salon}
\frame{
	\frametitle{Première console de salon!}

	\begin{minipage}[b]{0.48\linewidth}
		\centering
		Magnavox Odyssey\\
		(1972)\\
		\vspace{4em}
		\includegraphics[width=1.0\linewidth]{./figures/Magnavox_Odyssey_Logo.png}\\
		\includegraphics[width=1.0\linewidth]{./figures/Magnavox-Odyssey-Console-Set.jpg}
	\end{minipage}
	\hfill
	\begin{minipage}[b]{0.48\linewidth}
		\pause
		\begin{itemize}
			\item Inventée par \textbf{Ralph Baer}
			\item Numérique et programmable
			\item Sans microprocesseur
		\end{itemize}
		\pause
		\begin{itemize}
			\item 28 jeux
			\item 2 "manettes"
			\item Affiche 3 points et 1 ligne
			\item Nécessite des calques
		\end{itemize}
		\pause
		\includegraphics[width=1.0\linewidth]{./figures/Magnavox-Color-Screen-Overlays.jpg}
	\end{minipage}
}

\frame{
	\frametitle{Ralph H. Baer}

	\begin{minipage}[b]{0.48\linewidth}
		\onslide<1->{
			\begin{itemize}
				\item Né en allemagne en 1922
				\item États-Unis en 1938
				\item Technicien en 1940
				\item Enrolé en 1943
				\item Ingénieur en 1949
				\item Mort en 2014
			\end{itemize}
		}
		\onslide<2>{
		\begin{itemize}
				\item Développe l'Odyssey en indépendant
				\item Essaye de vendre le concept
					\begin{itemize}
						\item Fabriquants de téléviseurs
					\end{itemize}
				\item Contrat signé avec Magnavox entre 1970 et 1972
			\end{itemize}
		}
		\vspace{1em}
	\end{minipage}
	\hfill
	\begin{minipage}[b]{0.48\linewidth}
		\centering
		\onslide<1->{
			\includegraphics[width=.9\linewidth]{./figures/Ralph-Baer.jpg}
		}
	\end{minipage}
}

\frame{
	\frametitle{Magnavox Odyssey\href{http://www.digitpress.com/library/techdocs/o2-svc.pdf}{[1]}}
	\pause
	\centering
	\includegraphics[width=.9\linewidth]{./figures/magnavox_inside.png}
}

\section[8 générations]{8 générations de consoles}
\subsection[8 générations]{8 générations de consoles}
\frame{
	\frametitle{8 générations de consoles}
	\begin{minipage}[b][\textheight]{\textwidth}
		\begin{minipage}[b]{.68\linewidth}
		\begin{enumerate}
			\item Première génération
				\begin{itemize}
					\item Magnavox Odyssey (1972)
					\item Nintendo Color TV Game (1980)
				\end{itemize}
		\end{enumerate}
		\end{minipage}
		\hfill
		\begin{minipage}[b]{.28\linewidth}
			\includegraphics[width=.9\linewidth]{./figures/1gen.png}
		\end{minipage}
		\vfill
		\begin{center}
			\tiny
	%		red, green, blue, cyan , magenta, yellow, black, gray, darkgray, lightgray, brown, lime, olive, orange, pink, purple, teal, violet and white.
			\begin{tikzpicture}[xscale=1.0,yscale=1.0,x=0.22cm,y=0.22cm]
				\draw[color=gray] (1970,0) grid[xstep=2,ystep=18] (2020,18);
				\draw (1970,0) grid[xstep=10,ystep=18] (2020,18);
				\foreach \x in {1970,1980,1990,...,2020} \draw(\x,0)node[below]{\x};
				\draw[fill=lightgray] (1972,16.8) rectangle (1980,15);
				\draw[align=center] (1976,15.9) node{G1 (1972,1980)};
			\end{tikzpicture}
		\end{center}
		\vspace{2em}
	\end{minipage}
}

\frame{
	\frametitle{8 générations de consoles}
	\begin{minipage}[b][\textheight]{\textwidth}
		\begin{minipage}[b]{.68\linewidth}
		\begin{enumerate}
			\setcounter{enumi}{1}
			\item Deuxième génération
				\begin{itemize}
					\item Fairchild Channel F (1976)
					\item Krach du jeux vidéo de 1983
					\item Atari 2600 (1992)
				\end{itemize}
		\end{enumerate}
		\end{minipage}
		\hfill
		\begin{minipage}[b]{.28\linewidth}
			\includegraphics[width=.9\linewidth]{./figures/2gen.png}
		\end{minipage}
		\vfill
		\begin{center}
			\tiny
	%		red, green, blue, cyan , magenta, yellow, black, gray, darkgray, lightgray, brown, lime, olive, orange, pink, purple, teal, violet and white.
			\begin{tikzpicture}[xscale=1.0,yscale=1.0,x=0.22cm,y=0.22cm]
				\draw[color=gray] (1970,0) grid[xstep=2,ystep=18] (2020,18);
				\draw (1970,0) grid[xstep=10,ystep=18] (2020,18);
				\foreach \x in {1970,1980,1990,...,2020} \draw(\x,0)node[below]{\x};
				\draw[fill=lightgray] (1972,16.8) rectangle (1980,15);
				\draw[align=center] (1976,15.9) node{G1 (1972,1980)};
				\draw[fill=green] (1976,14.8) rectangle (1992,13);
				\draw[align=center] (1980,13.9) node{G2 (1976,1992)};
			\end{tikzpicture}
		\end{center}
		\vspace{2em}
	\end{minipage}
}

\frame{
	\frametitle{8 générations de consoles}
	\begin{minipage}[b][\textheight]{\textwidth}
		\begin{minipage}[b]{.68\linewidth}
		\begin{enumerate}
			\setcounter{enumi}{2}
			\item Troisième génération
				\begin{itemize}
					\item Post Krach de 83 
					\item Nintendo Entertainment System (NES) (1983)
					\item Nintendo Famicom (2003)
				\end{itemize}
		\end{enumerate}
		\end{minipage}
		\hfill
		\begin{minipage}[b]{.28\linewidth}
			\includegraphics[width=.9\linewidth]{./figures/3gen.png}
		\end{minipage}
		\vfill
		\begin{center}
			\tiny
	%		red, green, blue, cyan , magenta, yellow, black, gray, darkgray, lightgray, brown, lime, olive, orange, pink, purple, teal, violet and white.
			\begin{tikzpicture}[xscale=1.0,yscale=1.0,x=0.22cm,y=0.22cm]
				\draw[color=gray] (1970,0) grid[xstep=2,ystep=18] (2020,18);
				\draw (1970,0) grid[xstep=10,ystep=18] (2020,18);
				\foreach \x in {1970,1980,1990,...,2020} \draw(\x,0)node[below]{\x};
				\draw[fill=lightgray] (1972,16.8) rectangle (1980,15);
				\draw[align=center] (1976,15.9) node{G1 (1972,1980)};
				\draw[fill=green] (1976,14.8) rectangle (1992,13);
				\draw[align=center] (1980,13.9) node{G2 (1976,1992)};
				\draw[fill=orange] (1983,12.8) rectangle (2003,11);
				\draw[align=center] (1987,11.9) node{G3 (1983,2003)};
			\end{tikzpicture}
		\end{center}
		\vspace{2em}
	\end{minipage}
}
\frame{
	\frametitle{8 générations de consoles}
	\begin{minipage}[b][\textheight]{\textwidth}
		\begin{minipage}[b]{.68\linewidth}
		\begin{enumerate}
			\setcounter{enumi}{3}
			\item Quatrième génération
				\begin{itemize}
					\item NEC PC-Engine (1987)
					\item SNK Neo Geo (2004)
				\end{itemize}
		\end{enumerate}
		\end{minipage}
		\hfill
		\begin{minipage}[b]{.28\linewidth}
			\includegraphics[width=.9\linewidth]{./figures/4gen.png}
		\end{minipage}
		\vfill
		\begin{center}
			\tiny
	%		red, green, blue, cyan , magenta, yellow, black, gray, darkgray, lightgray, brown, lime, olive, orange, pink, purple, teal, violet and white.
			\begin{tikzpicture}[xscale=1.0,yscale=1.0,x=0.22cm,y=0.22cm]
				\draw[color=gray] (1970,0) grid[xstep=2,ystep=18] (2020,18);
				\draw (1970,0) grid[xstep=10,ystep=18] (2020,18);
				\foreach \x in {1970,1980,1990,...,2020} \draw(\x,0)node[below]{\x};
				\draw[fill=lightgray] (1972,16.8) rectangle (1980,15);
				\draw[align=center] (1976,15.9) node{G1 (1972,1980)};
				\draw[fill=green] (1976,14.8) rectangle (1992,13);
				\draw[align=center] (1980,13.9) node{G2 (1976,1992)};
				\draw[fill=orange] (1983,12.8) rectangle (2003,11);
				\draw[align=center] (1987,11.9) node{G3 (1983,2003)};
				\draw[fill=cyan] (1987,10.8) rectangle (2004,9);
				\draw[align=center] (1991,9.9) node{G4 (1987,2004)};
			\end{tikzpicture}
		\end{center}
		\vspace{2em}
	\end{minipage}
}

\frame{
	\frametitle{8 générations de consoles}
	\begin{minipage}[b][\textheight]{\textwidth}
		\begin{minipage}[b]{.68\linewidth}
		\begin{enumerate}
			\setcounter{enumi}{4}
			\item Cinquième génération
				\begin{itemize}
					\item 3DO Interactive Multiplayer (1993)
					\item Sony PSOne (2005)
				\end{itemize}
		\end{enumerate}
		\end{minipage}
		\hfill
		\begin{minipage}[b]{.28\linewidth}
			\includegraphics[width=.9\linewidth]{./figures/5gen.png}
		\end{minipage}
		\vfill
		\begin{center}
			\tiny
	%		red, green, blue, cyan , magenta, yellow, black, gray, darkgray, lightgray, brown, lime, olive, orange, pink, purple, teal, violet and white.
			\begin{tikzpicture}[xscale=1.0,yscale=1.0,x=0.22cm,y=0.22cm]
				\draw[color=gray] (1970,0) grid[xstep=2,ystep=18] (2020,18);
				\draw (1970,0) grid[xstep=10,ystep=18] (2020,18);
				\foreach \x in {1970,1980,1990,...,2020} \draw(\x,0)node[below]{\x};
				\draw[fill=lightgray] (1972,16.8) rectangle (1980,15);
				\draw[align=center] (1976,15.9) node{G1 (1972,1980)};
				\draw[fill=green] (1976,14.8) rectangle (1992,13);
				\draw[align=center] (1980,13.9) node{G2 (1976,1992)};
				\draw[fill=orange] (1983,12.8) rectangle (2003,11);
				\draw[align=center] (1987,11.9) node{G3 (1983,2003)};
				\draw[fill=cyan] (1987,10.8) rectangle (2004,9);
				\draw[align=center] (1991,9.9) node{G4 (1987,2004)};
				\draw[fill=magenta] (1993,8.8) rectangle (2005,7);
				\draw[align=center] (1997,7.9) node{G5 (1993,2005)};
			\end{tikzpicture}
		\end{center}
		\vspace{2em}
	\end{minipage}
}

\frame{
	\frametitle{8 générations de consoles}
	\begin{minipage}[b][\textheight]{\textwidth}
		\begin{minipage}[b]{.68\linewidth}
		\begin{enumerate}
			\setcounter{enumi}{5}
			\item Sixième génération
				\begin{itemize}
					\item Sega Dreamcast (1998)
					\item Sony PS2 (2013)
				\end{itemize}
		\end{enumerate}
		\end{minipage}
		\hfill
		\begin{minipage}[b]{.28\linewidth}
			\includegraphics[width=.9\linewidth]{./figures/6gen.png}
		\end{minipage}
		\vfill
		\begin{center}
			\tiny
	%		red, green, blue, cyan , magenta, yellow, black, gray, darkgray, lightgray, brown, lime, olive, orange, pink, purple, teal, violet and white.
			\begin{tikzpicture}[xscale=1.0,yscale=1.0,x=0.22cm,y=0.22cm]
				\draw[color=gray] (1970,0) grid[xstep=2,ystep=18] (2020,18);
				\draw (1970,0) grid[xstep=10,ystep=18] (2020,18);
				\foreach \x in {1970,1980,1990,...,2020} \draw(\x,0)node[below]{\x};
				\draw[fill=lightgray] (1972,16.8) rectangle (1980,15);
				\draw[align=center] (1976,15.9) node{G1 (1972,1980)};
				\draw[fill=green] (1976,14.8) rectangle (1992,13);
				\draw[align=center] (1980,13.9) node{G2 (1976,1992)};
				\draw[fill=orange] (1983,12.8) rectangle (2003,11);
				\draw[align=center] (1987,11.9) node{G3 (1983,2003)};
				\draw[fill=cyan] (1987,10.8) rectangle (2004,9);
				\draw[align=center] (1991,9.9) node{G4 (1987,2004)};
				\draw[fill=magenta] (1993,8.8) rectangle (2005,7);
				\draw[align=center] (1997,7.9) node{G5 (1993,2005)};
				\draw[fill=yellow] (1998,6.8) rectangle (2013,5);
				\draw[align=center] (2002,5.9) node{G6 (1998,2013)};
			\end{tikzpicture}
		\end{center}
		\vspace{2em}
	\end{minipage}
}

\frame{
	\frametitle{8 générations de consoles}
	\begin{minipage}[b][\textheight]{\textwidth}
		\begin{minipage}[b]{.68\linewidth}
		\begin{enumerate}
			\setcounter{enumi}{6}
			\item Septième génération
				\begin{itemize}
					\item XBox 360 (2005)
					\item PS3 (2017)
				\end{itemize}
		\end{enumerate}
		\end{minipage}
		\hfill
		\begin{minipage}[b]{.28\linewidth}
			\includegraphics[width=.9\linewidth]{./figures/7gen.png}
		\end{minipage}
		\vfill
		\begin{center}
			\tiny
	%		red, green, blue, cyan , magenta, yellow, black, gray, darkgray, lightgray, brown, lime, olive, orange, pink, purple, teal, violet and white.
			\begin{tikzpicture}[xscale=1.0,yscale=1.0,x=0.22cm,y=0.22cm]
				\draw[color=gray] (1970,0) grid[xstep=2,ystep=18] (2020,18);
				\draw (1970,0) grid[xstep=10,ystep=18] (2020,18);
				\foreach \x in {1970,1980,1990,...,2020} \draw(\x,0)node[below]{\x};
				\draw[fill=lightgray] (1972,16.8) rectangle (1980,15);
				\draw[align=center] (1976,15.9) node{G1 (1972,1980)};
				\draw[fill=green] (1976,14.8) rectangle (1992,13);
				\draw[align=center] (1980,13.9) node{G2 (1976,1992)};
				\draw[fill=orange] (1983,12.8) rectangle (2003,11);
				\draw[align=center] (1987,11.9) node{G3 (1983,2003)};
				\draw[fill=cyan] (1987,10.8) rectangle (2004,9);
				\draw[align=center] (1991,9.9) node{G4 (1987,2004)};
				\draw[fill=magenta] (1993,8.8) rectangle (2005,7);
				\draw[align=center] (1997,7.9) node{G5 (1993,2005)};
				\draw[fill=yellow] (1998,6.8) rectangle (2013,5);
				\draw[align=center] (2002,5.9) node{G6 (1998,2013)};
				\draw[fill=brown] (2005,4.8) rectangle (2017,3);
				\draw[align=center] (2009,3.9) node{G7 (2005,2017)};
			\end{tikzpicture}
		\end{center}
		\vspace{2em}
	\end{minipage}
}

\frame{
	\frametitle{8 générations de consoles}
	\begin{minipage}[b][\textheight]{\textwidth}
		\begin{minipage}[b]{.68\linewidth}
			\begin{enumerate}
				\setcounter{enumi}{7}
				\item Huitième génération
					\begin{itemize}
						\item Wii U (2012)
						\item Toujours en cours
					\end{itemize}
			\end{enumerate}
		\end{minipage}
		\hfill
		\begin{minipage}[b]{.28\linewidth}
			\includegraphics[width=.9\linewidth]{./figures/8gen.png}
		\end{minipage}
		\vfill
		\begin{center}
			\tiny
	%		red, green, blue, cyan , magenta, yellow, black, gray, darkgray, lightgray, brown, lime, olive, orange, pink, purple, teal, violet and white.
			\begin{tikzpicture}[xscale=1.0,yscale=1.0,x=0.22cm,y=0.22cm]
				\draw[color=gray] (1970,0) grid[xstep=2,ystep=18] (2020,18);
				\draw (1970,0) grid[xstep=10,ystep=18] (2020,18);
				\foreach \x in {1970,1980,1990,...,2020} \draw(\x,0)node[below]{\x};
				\draw[fill=lightgray] (1972,16.8) rectangle (1980,15);
				\draw[align=center] (1976,15.9) node{G1 (1972,1980)};
				\draw[fill=green] (1976,14.8) rectangle (1992,13);
				\draw[align=center] (1980,13.9) node{G2 (1976,1992)};
				\draw[fill=orange] (1983,12.8) rectangle (2003,11);
				\draw[align=center] (1987,11.9) node{G3 (1983,2003)};
				\draw[fill=cyan] (1987,10.8) rectangle (2004,9);
				\draw[align=center] (1991,9.9) node{G4 (1987,2004)};
				\draw[fill=magenta] (1993,8.8) rectangle (2005,7);
				\draw[align=center] (1997,7.9) node{G5 (1993,2005)};
				\draw[fill=yellow] (1998,6.8) rectangle (2013,5);
				\draw[align=center] (2002,5.9) node{G6 (1998,2013)};
				\draw[fill=brown] (2005,4.8) rectangle (2017,3);
				\draw[align=center] (2009,3.9) node{G7 (2005,2017)};
				\draw[fill=lime] (2011,2.8) rectangle (2017,1);
				\draw[align=center] (2014,1.9) node{G8 (2011,...)};
			\end{tikzpicture}
		\end{center}
		\vspace{2em}
	\end{minipage}
}

\end{document}


